Das Seminar ``Praktische Analyse räumlicher Daten in R'' orientierte sich zu einem großen Teil an ``Geocompuation with R'' \cite{lovelace2019}, einer detaillierten Einführung zum Import, der Exploration und der Analyse georeferenzierter Daten in R. Diese Arbeit stützt sich hauptsächlich auf die dort, in den Kapiteln 11 und 14, behandelten Methoden zur Anwendung statistischen Lernens auf geographische Daten. \\
Es werden Ergebnisse und Performance von logistischer Regression, einer Support Vector Machine und eines Random Forest Modells zur Klassifikation, anhand eines Datensatzes von eisenzeitlichen Fundstellen innerhalb Bayerns miteinander verglichen. \\
Der erste Abschnitt behandelt die Konstruktion des verwendeten Rasterdatensatzes auf Basis der Koordinaten archäologischer Fundstellen. Dabei werden auch verschiedene Techniken zum Errechnen minimaler Distanzen zu Gewässern miteinander verglichen um eine interaktive Visualisierung des zum Sampling von Pseudo-Non-Presence Datenpunkten zu ermöglichen. Danach werden alle Klassifikationsverfahren und Prozeduren um die Performance dieser zu schätzen vorgestellt. Im Anschluss daran werden erhaltene Ergebnisse anhand von predictive- und cell-difference-maps visualisiert und diskutiert. Zuletzt werden Resultate und Probleme bei der Interpretation dieser, sowie generelle Schwierigkeiten bei der Anwendung der vorgestellten Methoden und weiterführende Möglichkeiten zur Analyse besprochen. 