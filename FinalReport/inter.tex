Die angewendeten Methoden schienen für das Erstellen von Prognosekarten ideal, allerdings lassen die starke Streuung der spatial Vorhersageschätzer vermuten, dass das Forschungsareal zu groß gewählt sein könnte. Die predictive Maps selbst sollten  aus ökologischer bzw. archäologischer Sicht weniger als ``Chance'' einen Fund im entsprechenden Rasterpixel zu machen, sondern mehr als Index für die Habitateignung der Rasterzellen für Menschen aus der Eisenzeit interpretiert werden. Des Weiteren ist das kontrastreiche Ergebnis des Random Forest Modells unter Umständen zu bevorzugen, da dieses die konservativsten Vorhersagen macht. Aus archäologischer Sicht ist dies durchaus erwünscht, schließlich bringen Ausgrabungen große Kosten mit sich und sollten auf stichhaltigen Beweisen basieren. Insgesamt sind die Ergebniskarten mit Vorsicht zu genießen; die Auflösung von der benutzten Rasterdaten ist zu gering um wirklich sinnvolle Prognosen aufzustellen. Es bieten sich Folgeuntersuchungen an, die kleinere Teilareale Bayerns mit höherauflösenden Rasterdaten analysieren könnten. Abgesehen davon sollte das in Kapitel 2 angesprochene Problem zum Finden eines idealen Bufferradius zum Samplen von pseudo-nonpräsenz Punkten tiefergehend untersucht werden.